\usepackage{xcolor}
\usepackage{afterpage}
\usepackage{pifont,mdframed}
\usepackage[bottom]{footmisc}
\usepackage{minted}

\createsection{\Grader}{Grader di prova}
\newcommand{\inputfile}{\texttt{stdin}}
\newcommand{\outputfile}{\texttt{stdout}}
\makeatletter
\renewcommand{\this@inputfilename}{\texttt{stdin}}
\renewcommand{\this@outputfilename}{\texttt{stdout}}
\renewcommand{\this@syllabuslevel}{5}
\renewcommand{\this@custdifficulty}{3}
\makeatother
\setdifficulty{4}

% % % % % % % % % % % % % % % % % % % % % % % % % % % % % % % % % % % % % % % % % % %
% % % % % % % % % % % % % % % % % % % % % % % % % % % % % % % % % % % % % % % % % % %
L'allevatore Pierignolo, rivale di Abbondanzio, ha recentemente avuto una fuga di anatre dalla sua fattoria.
Gli animali si sono sentiti ispirati dal libro \textit{La fattoria degli animali di George Orwell} e hanno
deciso di evadere.

Per evitare che accada nuovamente, Pierignolo ha intenzione di costruire un recinto di \textit{Praseodimio(III) ossido}.

\begin{figure}[h]
    \centering
    \includegraphics[width=0.5\textwidth]{toxic.png}
    \caption{L'azienda che produce il \textit{Praseodimio(III) ossido}.}
\end{figure}

Il recinto deve essere lungo esattamente $K$ metri e Pierignolo ha solo un pezzo di \textit{Praseodimio(III) ossido} lungo $N$ metri.
Pierignolo può tagliare il pezzo come preferisce, grazie alla sua nuova smerigliatrice angolare.

Scopri se Pierignolo può creare il recinto! Scrivi \texttt{SI} se può crearlo, altrimenti scrivi \texttt{NO}.


\Implementation

Dovrai sottoporre un unico file, con estensione \texttt{.cpp}.

\begin{warning}
    Tra gli allegati a questo task troverai un template \texttt{recinto.cpp} con un esempio di implementazione.
\end{warning}

Il file di input è composto da $2$ righe:
\begin{itemize}
    \item Riga 1: gli interi $K$ e $N$, separati da uno spazio.
\end{itemize}

Il file di output è composto da $1$ riga:
\begin{itemize}
    \item Riga 1: la risposta al problema.
\end{itemize}

% % % % % % % % % % % % % % % % % % % % % % % % % % % % % % % % % % % % % % % % % % %
% % % % % % % % % % % % % % % % % % % % % % % % % % % % % % % % % % % % % % % % % % %

\Constraints

\begin{itemize}[nolistsep, itemsep=2mm]
    \item $1 \le K, N \le 1\:000\:000$.
\end{itemize}

% % % % % % % % % % % % % % % % % % % % % % % % % % % % % % % % % % % % % % % % % % %
% % % % % % % % % % % % % % % % % % % % % % % % % % % % % % % % % % % % % % % % % % %

\Scoring

Il tuo programma verrà testato su diversi test case raggruppati in subtask.
Per ottenere il punteggio relativo ad un subtask,
è necessario risolvere correttamente tutti i test che lo compongono.

\IIOTsubtask{0}{1}{Casi d'esempio.}

\IIOTsubtask{25}{1}{$K, N\le 30$.}

\IIOTsubtask{75}{1}{Nessuna limitazione aggiuntiva.}


% % % % % % % % % % % % % % % % % % % % % % % % % % % % % % % % % % % % % % % % % % %
% % % % % % % % % % % % % % % % % % % % % % % % % % % % % % % % % % % % % % % % % % %

\Examples

\begin{example}
    \exmpfile{recinto.input0.txt}{recinto.output0.txt}%
    \exmpfile{recinto.input1.txt}{recinto.output1.txt}%
    \exmpfile{recinto.input2.txt}{recinto.output2.txt}%
\end{example}

% % % % % % % % % % % % % % % % % % % % % % % % % % % % % % % % % % % % % % % % % % %
% % % % % % % % % % % % % % % % % % % % % % % % % % % % % % % % % % % % % % % % % % %

\Explanation

Nel primo caso d'esempio il pezzo di metallo non è abbastanza lungo.

Nei seguenti casi, il pezzo di metallo è abbastanza lungo per creare il recinto.