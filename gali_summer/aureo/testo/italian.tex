\usepackage{xcolor}
\usepackage{afterpage}
\usepackage{pifont,mdframed}
\usepackage[bottom]{footmisc}
\usepackage{minted}

\createsection{\Grader}{Grader di prova}
\newcommand{\inputfile}{\texttt{stdin}}
\newcommand{\outputfile}{\texttt{stdout}}
\makeatletter
\renewcommand{\this@inputfilename}{\texttt{stdin}}
\renewcommand{\this@outputfilename}{\texttt{stdout}}
\renewcommand{\this@syllabuslevel}{5}
\renewcommand{\this@custdifficulty}{3}
\makeatother
\setdifficulty{4}

% % % % % % % % % % % % % % % % % % % % % % % % % % % % % % % % % % % % % % % % % % %
% % % % % % % % % % % % % % % % % % % % % % % % % % % % % % % % % % % % % % % % % % %
Durante una lezione di matematica, Evan si imbatte nella sezione aurea, ovvero una costante definita nel seguente modo:
$$\Phi = \frac{1 + \sqrt{5}}{2}$$

Per i Greci, la sezione aurea rappresentava l'ideale di bellezza e armonia matematica nella natura e nell'arte. Il nome
"sezione aurea" deriva dal greco "\textit{hē chrysē tomē}", che significa "taglio d'oro", indicando il valore prezioso e perfetto
attribuito a questa proporzione.

\begin{figure}[h]
    \centering
    \includegraphics[width=0.5\textwidth]{aureo.png}
    \caption{La sezione aurea si può ritrovare ovunque.}
\end{figure}

Tuttavia Evan non è convinto della bellezza di questo numero, e perciò decide di inventarne uno nuovo, che si chiamerà per l'appunto
"numero di Evan" e verrà indicato con la lettera $\mathcal{E}$. Il numero di Evan non è una costante, bensì una funzione che ha come
argomento un array $A$ lungo $N$, ed è definita in questo modo:
$$\mathcal{E}(A) = f(1) \cdot f(2) \cdot … \cdot  f(N)$$
Mentre la funzione $f(i)$ è definita come segue:
\begin{itemize}
    \item Ordina i primi $i$ elementi dell'array $A$ in ordine non decrescente, dando origine ad un nuovo array $s_i$.
    \item $f(i) = |s_0\cdot 1 + s_1 \cdot 2+ … + s_{i-1} \cdot i|$
\end{itemize}

Ad esempio, se $A = [2,1,4]$:
\begin{itemize}
    \item $s_1 = [2]$, quindi $f(1) = |2\cdot 1 |= 2$
    \item $s_2 = [1,2]$, quindi $f(2) = |1\cdot 1 + 2\cdot 2| = 5$
    \item $s_3 = [1,2,4]$, quindi $f(3) = |1\cdot 1 + 2\cdot 2 + 4\cdot 3|= 17$
\end{itemize}

Di conseguenza $\mathcal{E}(A) = f(1) \cdot f(2) \cdot f(3) = 2 \cdot 5 \cdot 17 = 170$

Ora è giunto il momento di testare la validità di questo numero, perciò dato un array $A$ lungo $N$,
dovrai calcolare $\mathcal{E}(A)$. Dato che può essere enorme, dovrai calcolare $\mathcal{E}(A)\mod (10^9+7)$.


\Implementation

Dovrai sottoporre un unico file, con estensione \texttt{.cpp}.

\begin{warning}
    Tra gli allegati a questo task troverai un template \texttt{aureo.cpp} con un esempio di implementazione.
\end{warning}

Il file di input è composto da $2$ righe:
\begin{itemize}
    \item Riga 1: l'intero $N$.
    \item Riga 2: $N$ interi che compongo l'array $A$.
\end{itemize}

Il file di output è composto da $1$ riga:
\begin{itemize}
    \item Riga 1: la risposta al problema.
\end{itemize}

% % % % % % % % % % % % % % % % % % % % % % % % % % % % % % % % % % % % % % % % % % %
% % % % % % % % % % % % % % % % % % % % % % % % % % % % % % % % % % % % % % % % % % %

\Constraints

\begin{itemize}[nolistsep, itemsep=2mm]
    \item $1 \le N \le 100\:000$.
    \item $-10^5 \le A_i \le 10^5$ per ogni $i = 0 \dots N-1$.
\end{itemize}

% % % % % % % % % % % % % % % % % % % % % % % % % % % % % % % % % % % % % % % % % % %
% % % % % % % % % % % % % % % % % % % % % % % % % % % % % % % % % % % % % % % % % % %

\Scoring

Il tuo programma verrà testato su diversi test case raggruppati in subtask.
Per ottenere il punteggio relativo ad un subtask,
è necessario risolvere correttamente tutti i test che lo compongono.

\IIOTsubtask{0}{1}{Casi d'esempio.}

\IIOTsubtask{20}{1}{$N \le 100$ e $V_i >= 0$ per ogni $i = 0\dots N$.}

\IIOTsubtask{40}{1}{$N \le 100$.}

\IIOTsubtask{25}{4}{$V_i >= 0$ per ogni $i = 0\dots N$.}

\IIOTsubtask{15}{4}{Nessuna limitazione aggiuntiva.}


% % % % % % % % % % % % % % % % % % % % % % % % % % % % % % % % % % % % % % % % % % %
% % % % % % % % % % % % % % % % % % % % % % % % % % % % % % % % % % % % % % % % % % %

\Examples

\begin{example}
    \exmpfile{aureo.input0.txt}{aureo.output0.txt}%
    \exmpfile{aureo.input1.txt}{aureo.output1.txt}%
    \exmpfile{aureo.input2.txt}{aureo.output2.txt}%
\end{example}

% % % % % % % % % % % % % % % % % % % % % % % % % % % % % % % % % % % % % % % % % % %
% % % % % % % % % % % % % % % % % % % % % % % % % % % % % % % % % % % % % % % % % % %

\Explanation

Il primo caso d'esempio è quello descritto dal problema.

Nel secondo caso d'esempio:
\begin{itemize}
    \item $f(1) = |2\cdot 1| = 2$
    \item $f(2) = |-5\cdot 1 + 2\cdot 2| = 1$
    \item $f(3) = |-5\cdot 1 + 2\cdot 2 + 2\cdot 5| = 5$
\end{itemize}
Quindi $\mathcal{E}(A) = 2 \cdot 1 \cdot 5 = 10$.
