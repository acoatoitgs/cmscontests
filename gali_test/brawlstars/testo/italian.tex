\usepackage{xcolor}
\usepackage{afterpage}
\usepackage{pifont,mdframed}
\usepackage[bottom]{footmisc}
\usepackage{minted}

\createsection{\Grader}{Grader di prova}

\newcommand{\inputfile}{\texttt{stdin}}
\newcommand{\outputfile}{\texttt{stdout}}
\makeatletter
\renewcommand{\this@inputfilename}{\texttt{stdin}}
\renewcommand{\this@outputfilename}{\texttt{stdout}}
\makeatother

% % % % % % % % % % % % % % % % % % % % % % % % % % % % % % % % % % % % % % % % % % %
% % % % % % % % % % % % % % % % % % % % % % % % % % % % % % % % % % % % % % % % % % %

Durante la lezione di italiano, mentre l'insegnante spiegava \textit{"L'infinito" di Giacomo Leopardi}, il tuo amico Carlo stava giocando a Brawl Stars.
Purtroppo però, dopo aver esultato ad alta voce per aver vinto una partita, l'insegnante si è accorta della sua mancanza di attenzione e
ha deciso di ritiragli il telefono.

\begin{figure}[h]
    \centering
    \includegraphics[width=0.5\textwidth]{brawl.jpg}
    \caption{Il telefono di Carlo, momenti prima del disastro. }
\end{figure}

Per fargli imparare la lezione, ha deciso di dargli $T$ esercizi di matematica\footnote{Il passatempo preferito dell'insegnante è risolvere problemi di informatica su \url{https://training.olinfo.it}},
ognuno composto da 2 interi, $A_i \leq B_i$. L'obiettivo è quello di trovare la somma dei numeri compresi tra $A_i$ e $B_i$, ovvero
$A_i + (A_i + 1) + (A_i+2) + \dots + B_i$. Dato che questa somma può essere enorme, il risultato va espresso \textbf{modulo 1000003}. Aiuta Carlo a risolvere i tediosi esercizi dell'insegnante.

\begin{danger}
    Per evitare di andare in overflow durante i calcoli intermedi, è necessario sfruttare l'aritmetica
    modulare:
    \begin{itemize}
        \item (a + b) \% m == ((a \% m) + (b \% m)) \% m
        \item (a * b) \% m == ((a \% m) * (b \% m)) \% m
    \end{itemize}

    Oppure svolgi tutti i calcoli usando una variabile di tipo \texttt{long long} e restituisci il risultato modulo $1000003$.
\end{danger}
% % % % % % % % % % % % % % % % % % % % % % % % % % % % % % % % % % % % % % % % % % %
% % % % % % % % % % % % % % % % % % % % % % % % % % % % % % % % % % % % % % % % % % %

\Implementation

Dovrai sottoporre un unico file, con estensione \texttt{.cpp}.

\begin{warning}
    Tra gli allegati a questo task troverai un template \texttt{brawlstars.cpp} con un esempio di implementazione.
\end{warning}

Il file di input è composto da $T+1$ righe:
\begin{itemize}
    \item Riga 1: l'intero T.
    \item Riga 2 \dots T+1: gli interi $A_i$ e $B_i$.
\end{itemize}

Il file di output è composto da $1$ riga:
\begin{itemize}
    \item Riga 1: le risposte agli esercizi in ordine, separate da uno spazio.
\end{itemize}

% % % % % % % % % % % % % % % % % % % % % % % % % % % % % % % % % % % % % % % % % % %
% % % % % % % % % % % % % % % % % % % % % % % % % % % % % % % % % % % % % % % % % % %

\Constraints

\begin{itemize}[nolistsep, itemsep=2mm]
    \item $1 \le T \le 10000$.
    \item $1 \leq A_i \leq B_i \leq 10^9$.
\end{itemize}

% % % % % % % % % % % % % % % % % % % % % % % % % % % % % % % % % % % % % % % % % % %
% % % % % % % % % % % % % % % % % % % % % % % % % % % % % % % % % % % % % % % % % % %

\Scoring

Il tuo programma verrà testato su diversi test case raggruppati in subtask.
Per ottenere il punteggio relativo ad un subtask,
è necessario risolvere correttamente tutti i test che lo compongono.

\begin{itemize}[nolistsep,itemsep=2mm]
    \item \subtask Casi d'esempio.
    \item \subtask $A_i = 1$ e $B_i \leq 10000$.
    \item \subtask $A_i = 1$ e $B_i \leq 10000000$.
    \item \subtask $A_i = 1$.
    \item \subtask $B_i \leq 10000$.
    \item \subtask $B_i \leq 10000000$.
    \item \subtask Nessuna limitazione aggiuntiva.
\end{itemize}

% % % % % % % % % % % % % % % % % % % % % % % % % % % % % % % % % % % % % % % % % % %
% % % % % % % % % % % % % % % % % % % % % % % % % % % % % % % % % % % % % % % % % % %

\Examples

\begin{example}
    \exmpfile{brawlstars.input0.txt}{brawlstars.output0.txt}%
\end{example}

% % % % % % % % % % % % % % % % % % % % % % % % % % % % % % % % % % % % % % % % % % %
% % % % % % % % % % % % % % % % % % % % % % % % % % % % % % % % % % % % % % % % % % %

\Explanation

Nel caso d'esempio, c'è un solo esercizio. La somma dei numeri da $5$ a $7$ è $5+6+7 = 18$.
