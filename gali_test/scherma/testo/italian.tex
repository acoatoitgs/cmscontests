\usepackage{xcolor}
\usepackage{afterpage}
\usepackage{pifont,mdframed}
\usepackage[bottom]{footmisc}
\usepackage{minted}

\createsection{\Grader}{Grader di prova}

\newcommand{\inputfile}{\texttt{stdin}}
\newcommand{\outputfile}{\texttt{stdout}}
\makeatletter
\renewcommand{\this@inputfilename}{\texttt{stdin}}
\renewcommand{\this@outputfilename}{\texttt{stdout}}
\makeatother

% % % % % % % % % % % % % % % % % % % % % % % % % % % % % % % % % % % % % % % % % % %
% % % % % % % % % % % % % % % % % % % % % % % % % % % % % % % % % % % % % % % % % % %

Kenan, amico di Noemi, è un fortissimo schermidore\footnote{Giocatore di scherma.}. Talmente forte, da
essere sicuro di vincere qualsiasi torneo.
In particolare, nel prossimo mese ci saranno $N$ tornei diversi, ognuno con un premio al vincitore di $P_i$ euro.

\begin{figure}[h]
    \centering
    \includegraphics[width=0.5\textwidth]{kenen.jpg}
    \caption{Kenen durante un suo scontro.}
\end{figure}
Tuttavia, Kenan ha un livello $E$ di energia, ed ogni torneo gli consumerà $L_i$ punti di energia. Nel caso l'energia rimasta fosse
inferiore di quella richiesta da un torneo, Kenan non potrà prenderne parte.
Trova quanti soldi può vincere Kenan se sceglie in modo ottimale i tornei a cui partecipare.





\Implementation

Dovrai sottoporre un unico file, con estensione \texttt{.cpp}.

\begin{warning}
    Tra gli allegati a questo task troverai un template \texttt{scherma.cpp} con un esempio di implementazione.
\end{warning}

Il file di input è composto da $N+1$ righe:
\begin{itemize}
    \item Riga 1: gli interi $N$ e $E$, separati da uno spazio.
    \item Riga 2 \dots N+1: gli interi $P_i$ e $L_i$.
\end{itemize}

Il file di output è composto da $1$ riga:
\begin{itemize}
    \item Riga 1: la risposta al problema.
\end{itemize}

% % % % % % % % % % % % % % % % % % % % % % % % % % % % % % % % % % % % % % % % % % %
% % % % % % % % % % % % % % % % % % % % % % % % % % % % % % % % % % % % % % % % % % %

\Constraints

\begin{itemize}[nolistsep, itemsep=2mm]
    \item $1 \le N \le 1\:000\:000$.
    \item $1 \le E \le 1\:000\:000$.
    \item $1 \le N \cdot E \le 3\:000\:000$.
    \item $1 \le P_i \le 1\:000\:000\:000$ per ogni $i = 0 \dots N-1$.
    \item $0 \le L_i \le E$ per ogni $i = 0 \dots N-1$.
    \item Kenan può partecipare ad ogni torneo una volta sola.
\end{itemize}

% % % % % % % % % % % % % % % % % % % % % % % % % % % % % % % % % % % % % % % % % % %
% % % % % % % % % % % % % % % % % % % % % % % % % % % % % % % % % % % % % % % % % % %

\Scoring

Il tuo programma verrà testato su diversi test case raggruppati in subtask.
Per ottenere il punteggio relativo ad un subtask,
è necessario risolvere correttamente tutti i test che lo compongono.

\begin{itemize}[nolistsep,itemsep=2mm]
    \item \subtask Casi d'esempio.
    \item \subtask $N \cdot E \le 10$.
    \item \subtask $L_i$ è uguale per tutti gli $i = 0 \dots N-1$.
    \item \subtask $N \le 20$.
    \item \subtask $N \cdot E \le 1\:000$.
    \item \subtask Nessuna limitazione aggiuntiva.
\end{itemize}

% % % % % % % % % % % % % % % % % % % % % % % % % % % % % % % % % % % % % % % % % % %
% % % % % % % % % % % % % % % % % % % % % % % % % % % % % % % % % % % % % % % % % % %

\Examples

\begin{example}
    \exmpfile{scherma.input0.txt}{scherma.output0.txt}%
\end{example}

% % % % % % % % % % % % % % % % % % % % % % % % % % % % % % % % % % % % % % % % % % %
% % % % % % % % % % % % % % % % % % % % % % % % % % % % % % % % % % % % % % % % % % %

\Explanation

Nel caso d'esempio a Kenan conviene partecipare a tutti e tre i tornei, ottenendo in totale $29$ euro.
